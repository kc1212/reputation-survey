% TODO expand, better intro
In this section we categorise various defence techniques against the
sybil-attack in reputation systems.

\subsection{Trusted Third Party}\label{sec:trusted_party}
One of the earliest and best known reputation system is
eBay\cite{resnick2002trust}. The buyers and sellers rely on a trusted third
party, in this case eBay, to gather and distribute feedbacks after every
transaction. Even when there are no incentives to provide feedback, Resnick and
Zeckhauser observed that feedback was provided more than half of the
time\cite{resnick2002trust}, making eBay one of the most well-known online
marketplaces.

In general, trusted third parties manage the issurance and verification of
identities. Thus they can apply a fee on the peer for creating a new
identity\cite{resnick2001social} or rate-limit the creation of new
identities\cite{douceur2002sybil}, making sybil-attacks more difficult.
Furthermore, trusted third parties often have the ability to manipulate the
identities. For example they could punish the attackers by disabling all of
their identity when caught, making the sybil-attack much riskier especially when
identities are costly.

Trusted third party is likely the most widely used technique in practice.
Marketplaces such as Amazon or eBay, online forums such as Stackoverflow or
Reddit, all use a form of trusted third party.

Unfortunately, a trusted third party is often a single point of failure.
Moreover, being a centralised system, it is difficult to scale up to suit
increasing user demands. % TODO GIVE exmaples of failures
In the remainder of this section, we focus on distributed techniques for
preventing the sybil-attack.

Credence 06\cite{walsh2006experience} - uses central authority to sign key

\subsection{Costly Identity Creation}\label{sec:costly_id}
\subsubsection{IP Address}
\subsubsection{Low reputation for new users}
Feldman 04\cite{feldman2004robust} - adaptive stranger, low score on entry

\subsection{Indirect Information}
EigenTrust\cite{kamvar2003eigentrust} - doesn't prevent sybils, suggests to add cost in ID creation
R2Trust\cite{tian2011h2trust} - credibility, tackles colluders, time decay factor

\subsection{Graph Techniques}\label{sec:graph}
Theory\cite{seuken2011sybil}
Gal-Oz et al. \cite{gal2008robust} communities are collection of knots, sybils can form a knot?
Regret\cite{sabater2001regret, sabater2002social} - information from multiple dimensions
Guha 04\cite{guha2004propagation} - no mention of sybil attacks or attacks in general

\subsubsection{Flow Based}
BarterCast\cite{meulpolder2009bartercast}
SybilRes\cite{delaviz2012sybilres}

\subsubsection{Topology}
SybilGuard\cite{yu2006sybilguard}
SybilLimit\cite{yu2008sybillimit}
SybilInfer\cite{danezis2009sybilinfer}
SybilShield\cite{shi2013sybilshield} - assuming sybils have bad connectedness
SumUp\cite{tran2009sybil}
GateKeeper\cite{tran2011optimal} - based on SumUp
Social-network\cite{viswanath2010analysis} - community detection

Distributed Sparse Cut Monitoring\cite{kurve2011sybil}

Other systems are built on top:
ReDS\cite{akavipat2014reds} suggests to use sybilimit or sybilinfer
SybilProof-DHT\cite{lesniewski2010whanau}
\subsection{Reputation Transfer}
Trust-transfer\cite{seigneur2005trust}

\subsection{Self Registration}
P-GRID 01\cite{aberer2001p}
Self-registration\cite{dinger2006defending} - distributed registration based on IP address

\subsection{Cryptography Based Techniques?}
Secure-Overlay\cite{lua2007securing} - ID crypto and SSS
Privacy-preserving\cite{schaub2016trustless} - blockchain?
Proof-of-stake\cite{dennis2016rep}
SybilConf\cite{tegeler2010sybilconf}

\subsection{Content Driven}
\cite{chatterjee2008robust}

\subsection{Other}
Parental control\cite{tehale2012parental} - uses parents to ``observe'' find suspects, only for detection, requires a sybil-proof reputation scheme
DSybil\cite{yu2009dsybil} - recommendation system, need historical data
Symon\cite{jyothi2009symon} - pair peers together, likelihood for both to be sybils is low, the pair monitor each other to prevent attacks
XRep 02\cite{damiani2002reputation} IP check, and checks digest, uses existing P2P systems like Gnutella

\subsection{Unsorted?}

Beth and PGP limits Sybil attack to some extent by using social graphs
Beth 94\cite{beth1994valuation}
PGP (Zimmermann) 95\cite{zimmermann1995official}

Yu 00\cite{yu2000social}
% CORE 02\cite{michiardi2002core} % MANETs
Lee 03\cite{lee2003cooperative} - uses flooding, might not be scalable, only talks about DoS
% Buchegger 04 - MANETs
% Xiong 03\cite{xiong2003reputation} % also PeerTrust?
Marti 04\cite{marti2004limited}
ARA 05\cite{ham2005ara} - no mention of sybil, prevents freeriding, prevents short-term abuse because reputation increases gradually
FuzzyTrust Song 05\cite{song2005trusted} - uses fuzzy logic
P2PRep/Fuzzy 06\cite{aringhieri2006fuzzy} - also fuzzy, does not prevent generation of false rumors
Xiong 05\cite{xiong2007countering} - no mention of sybil, but tries to mitigate false information
PowerTrust 06\cite{zhou2007powertrust} - uses ``power nodes'' (from power-law), no mention of sybil, some defence against colluders
% Li 07 - MANETs

Histos and Sopras\cite{zacharia2000collaborative}, doesn't really have structure?
Beta\cite{jsang2002beta}
% Confidant\cite{buchegger2002performance} MANETs
Gupta et al.\cite{gupta2003reputation}

PeerTrust\cite{xiong2004peertrust} - DHT, used P-GRID source code, has credibility rating

PerContRep\cite{yan2014percontrep}


\subsection{Does not handle Sybil-attack?}
TrustMe\cite{singh2003trustme} is a reputation that focuses on anonymity, no mention of sybil attack

H-Trust\cite{zhao2009htrust} does not mention sybil

Coner et al.\cite{conner2009trust} assumes clients cannot perform sybil attack

TrustGuard 05\cite{srivatsa2005trustguard} - assumes it is built on secure overlay networks (sybil-proof networks)

Scrivener 05\cite{nandi2005scrivener} - assumes ID cannot be created and discarded

%%% Local Variables:
%%% mode: latex
%%% TeX-master: "main"
%%% End:
