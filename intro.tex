
\begin{comment}
Electronic commerce and online social networks allow us to orchestrate many
aspects of our lives in the comfort of our homes, behind the monitors of our
devices. With such services, online identity is often required. For example in
Twitter\footnote{Twitter (\texttt{https://twitter.com/}) is a social network website
  where users can send and read short messages---called ``tweets'', users can
  also ``follow'' each other.}, users must create accounts to tweet a friend,
who must also have an account. In this scenario, users can choose to remain
pseudonymous if they are careful, where their real-life identity is uncorrelated
with their online identity. While that is useful for protecting the users'
privacy, it also opens an alleyway for attackers.
\end{comment}

A recent article in the Atlantic describes how fake Twitter accounts are shaping
the 2016 US presidential election~\cite{atlantictwitterbots}. Over a third of
pro-Trump tweets and almost a fifth of pro-Clinton tweets, totalling at about 1
million, came from bots. These cases date back to 2010. In the 2010
Massachusetts senate race, researchers found evidence that Republican
campaigners also used fake accounts to manipulate Google real-time search
results to tip in their favour, effectively causing a spread of
misinformation~\cite{mustafaraj2010obscurity}. These are examples of the sybil
attacks and they are a threat to democracy because opinions of real users are
eclipsed by an overwhelming number of fake accounts.

The sybil attack, first described by Douceur~\cite{douceur2002sybil}, is an
attack where an entity can assume multiple identities or sybils\footnote{In this
  work we use ``sybil'' as a noun to mean a fake identity.}, and then attack
either another entity or undermine the whole system. For example, a malicious
Twitter user can create many fake identities and have them follow his real
identity, thus creating a false reputation. It is one of the most important
attacks because it leads to numerous consequences including but not limited to
spreading false information, identity theft~\cite{bilge2009all} and ballot
stuffing~\cite{bhattacharjee2005avoiding}. Furthermore, to the best of our
knowledge, there is no general solution for preventing the sybil attack.

There has been over a decade of work on the sybil attack from both
perspective---attackers and defenders. Many previous surveys focus only on the
defender's perspective or a particular class of defence
mechanisms~\cite{marti2006taxonomy, mohaisen2013sybil, rakesh2014survey,
  koll2014state}. The purpose of this work is to survey and provide a broad view
on all aspects of the sybil attack. This work should be seen as an introductory
material for readers who wish to familiarise themselves with research in this
are.

This survey is organised as follows. First, we illustrate the importance of the
sybil attack by looking at how researchers and black-hat hackers mounted the
attack in the real-world; we also experiment with fake Twitter followers and
look at their properties (\autoref{sec:motivation}). Next, we give the models
and definitions that we use in this work and some important theoretical results
(\autoref{sec:theory}). With the fundamentals in place, we describe the various
types of attacks (\autoref{sec:attacks}) and defence mechanisms
(\autoref{sec:defences}).
% There is a large variety of sybil attack
% defence mechanisms, from using trusted-third-party to exploiting the graph
% characteristics in online social networks, thus we classify these mechanisms by
% their ``main idea''.
Finally, we present the related work in \autoref{sec:related} and conclude in
\autoref{sec:summary}.

%%% Local Variables:
%%% mode: latex
%%% TeX-master: "main"
%%% End:
