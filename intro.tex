Reputation systems (described in \autoref{sec:reputation}) allow entities,
usually humans, to trust each other in the cyberspace based on their prior
interactions (logical) or knowledge from other entities. For instance, online
marketplaces such as Amazon or eBay often use a reputation system, and new
buyers are more likely to buy goods from merchants with a high rating (a metric
for reputation).

However, reputation systems are vulnerable to many types of attacks. The
Sybil-attack, first described by Douceur\cite{douceur2002sybil}, is an attack
where an entity can assume multiple identities or Sybils, and then attack either
another entity or undermine the whole reputation system (we discuss it in more
details in \autoref{sec:sybil}). In the marketplace example, the merchant could
create multiple fake accounts and submitting a lot of positive feedback to the
real account to boost the rating. It is one of the most important attacks
because it leads to a large number of consequences including but not limited to
spreading false information, ballot stuffing\cite{bhattacharjee2005avoiding} and
eclipse attacks\cite{singh2006eclipse}. Thus, preventing the Sybil-attack is
likely to significantly increase the credibility of reputation systems.

Sybil-defence mechanisms come in various shapes and sizes. Some rely on a
trusted third party (\autoref{sec:trusted_party}), some introduce a cost in
identity creation (\autoref{sec:costly_id}), some exploit the graph
characteristics (\autoref{sec:graph}) and so on. To the best our knowledge,
there does not exist a recent and comprehensive survey that focuses on the
Sybil-attack in reputation systems.

% to this end?
% scope of this work
To this end, we survey the defence mechanisms proposed by various reputation
systems to eliminate or minimise Sybil-attacks as well as general approaches
that do not depend on any specific reputation systems. Note that Sybil-attacks
do not only exist in reputation systems. Wireless sensor networks for example
are also vulnerable, the attacker can cripple the routing algorithm or defeat
distributed storage mechanisms\cite{newsome2004sybil}. Thus defence mechanisms
that do not apply to reputation systems are outside the scope of this work and
are not covered. On the other hand, since reputation systems are often also
peer-to-peer systems, we do cover the more general defence mechanisms.

Our main contributions are the following.
\begin{enumerate}
  \item TODO
  \item TODO
\end{enumerate}

%%% Local Variables:
%%% mode: latex
%%% TeX-master: "main"
%%% End:
