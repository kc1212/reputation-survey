In this work, we survey both the practical and the theoretical aspects of the
sybil attack. We demonstrate the severeness of the sybil attack by showing the
harm it is causing in the real world. Social networks are flooded with sybils
making real news and fake news almost indistinguishable. Users of Tor are
monitored by sybils hidden in the network trying to reveal their real identity.
Then we define the sybil attack using its original definition from
Douceur~\cite{douceur2002sybil} and introduce one of the most common
models---the social graph. Next, we zoom into five different systems---online
social networks, file sharing networks, reputation systems, wireless ad-hoc
networks and the Bitcoin network---and look the various types of attacks that
can be mounted. There are a number of alarming attacks, for instance automated
identity theft, self-promotion and so on. The main part of our work summarises
the defence mechanisms. Earlier defence mechanisms primarily work by limiting
the number of identities or the rate at which they are created. The introduction
of BarterCast inspired many network-flow based techniques for limiting the
influence of sybils. Similarly, the introduction of SybilGuard stimulated a lot
of work on random-walk based techniques for identifying sybils. Hybrids are also
available, for instance GateKeeper is a hybrid of the two aforementioned
techniques. Finally, we compare and contrast our work with existing surveys.

We hope this work demonstrates the alarming consequences of the sybil attack and
many ingenious ways to defend against it. However, there does not exist a
general solution and many defence mechanisms must satisfy their own set of
assumptions in order to perform well. When the assumptions are violated, which
can be the case due to the dynamic structure of real networks, they become
ineffective (demonstrated in~\cite{liu2016smartwalk}). Moreover, almost no
defence mechanisms considers the temporal dynamics~\cite{liu2015exploiting},
i.e. the attacker may modify the attack edges or its the social graph in the
sybil region over time. Lin et el. show the attacker can ``greatly undermine the
security guarantees'' of many defence mechanisms~\cite{liu2015exploiting}.

Without a doubt, much work still needs to be done in order for the cyberspace to
be free of the sybil attack. We hope this work serves as a cornerstone for the
future defence mechanisms.

%%% Local Variables:
%%% mode: latex
%%% TeX-master: "main"
%%% End:
