In this work, we survey both the practical and the theoretical aspects of the
Sybil attack. We motivate the Sybil attack by showing the harm it is causing in
the real world. Social networks are flooded with Sybils making real news and
fake news almost indistinguishable. Users of Tor are monitored by Sybils hidden
in the network trying to reveal their real identity. Then we define the Sybil
attack using its original definition from Douceur~\cite{douceur2002sybil} and
introduce one of the most common models, i.e. the social graph. Next, we zoom
into four different systems---P2P file sharing networks, online social networks,
reputation systems and wireless ad-hoc networks---and look the various types of
attacks that can be mounted. There are a number of alarming attacks, for
instance automated identity theft, self-promotion and so on. The main part of
our work summarises the defence mechanisms. Earlier defence mechanisms primarily
work by limiting the number of identities or the rate at which they are created.
The introduction of BarterCast inspired many network-flow based techniques for
limiting the influence of Sybils. Similarly, the introduction of SybilGuard
stimulated a lot of work on random-walk based techniques for identifying Sybils.
Hybrids are also available, for instance GateKeeper is a hybrid of the two
aforementioned techniques. Finally, we compare and contrast our work with
existing surveys.

We hope this work demonstrates the alarming consequences of the Sybil attack and
many ingenious ways to defend against it. However, there does not exist a
general solution and many defence mechanisms must satisfy their own set of
assumptions in order to perform well. When the assumptions are violated, which
can be the case due to the dynamic structure of real networks, they become
ineffective (demonstrated in~\cite{liu2016smartwalk}). Moreover, almost no
defence mechanisms considers the temporal dynamics~\cite{liu2015exploiting},
i.e. the attacker may modify the attack edges or its the social graph in the
Sybil region over time. The authors show the attacker can ``greatly undermine
the security guarantees'' of many defence mechanisms.

Without a doubt, much work still needs to be done in order for the cyberspace to
be free of the Sybil attack. We hope this work serves as a cornerstone for the
future defence mechanisms.

%%% Local Variables:
%%% mode: latex
%%% TeX-master: "main"
%%% End:
