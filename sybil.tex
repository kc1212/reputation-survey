The Sybil attack is coined by Douceur\cite{douceur2002sybil} in 2002 in the
context of peer-to-peer systems. In this section, we first introduce the Sybil
attack using Douceur's original definition and outline the key (discouraging)
theoretical results. Next, we review practical attacks in three types of systems
(1) MANET (mobile ad-hoc networks) such as sensor networks, (2) reputation
systems such as PageRank\cite{page1999pagerank} but also include e-commerce
systems such as eBay and (3) OSN (online social networks) such as Twitter and
Facebook. We hope our review illuminates the alarming consequences of the Sybil
attack.

\subsection{Theoretical Results}
Douceur defined the Sybil attack as forging multiple identities under the same
entity\cite{douceur2002sybil}. An entity can be for example a physical user of
the system and identities are how entities present themselves to the system.
Thus a local entity has no direct knowledge of remote entities, only their
identities. The forged identities do not necessarily follow the protocol
specified by the underlying network, they may deviate arbitrarily from the
protocol, i.e. they assume the characteristics of Byzantine
fault\cite{lamport1982byzantine}. We use these terms in the remainder of the
survey.

The author modelled the system as a general distributed computing
environment where there is no constraint on the topology, every node has limited
computational resources and messages are guaranteed to be delivered. Under this
model, the author proved that the Sybil attack is always possible without a
central, trusted authority.

% Preventing the Sybil attack is in fact a lot
% more difficult because peer-to-peer systems often do not have a central, trusted
% authority.

Cheng and Friedman proved an important result regarding the Sybil attack in
reputation systems\cite{cheng2005sybilproof}. Reputation systems are commonly
used in MANET, e-commerce and the internet in general, where entities are
rewarded by their good behaviour and penalised otherwise. Google's
PageRank\cite{page1999pagerank} is an example of a reputation system, where a
large number of links to a website makes it more reputable. Cheng and Friedman
classified reputation systems into two categories,
\begin{enumerate}
\item symmetric reputation systems where the reputation score only depends on
  the network topology, popular reputation mechanisms such as
  PageRank\cite{page1999pagerank} and EigenTrust\cite{kamvar2003eigentrust} are
  examples of symmetric reputation systems, and
    \item asymmetric reputation systems where there some nodes are trusted and
      reputation scores are propagated through the trusted nodes, most OSN are
      examples of asymmetric reputation systems.
\end{enumerate}
The authors formally proved that symmetric reputation systems are vulnerable to
the Sybil attack. But in the asymmetric case, it is possible to construct a
Sybil-proof reputation system.

\subsection{The Sybil Attack in Online Social Networks}
OSN (online social networks) are vulnerable to the Sybil attack even when most
of them use a central, trusted authority such as Facebook. In OSN, users create
profiles and form relationships with friends. In contrast with real world
relationships, it is much easier to create relationships in OSN even with
strangers. Authors of \cite{sophos} created a Facebook profile and send friend
requests to 200 random users, and 41\% of the users accepted the friend request.
Combining with the ability to create new identities with very little cost, it is
possible to perform many types of attacks which we outline below.

\subsubsection{Identity Theft}
Authors of \cite{bilge2009all} created two attacks on five social
network sites including Facebook and LinkedIn - profile cloning and cross-site
profile cloning. The iCloner, was created to automate these attacks.

In profile cloning, iCloner used publicly available
information to automatically create clones of the victim's profiles, effectively
creating Sybils. iCloner then sent friend requests from the cloned profile
to the friends of the victim. The fact that the victim may have many
friends that they do not contact very often, e.g. friend from primary school
living in another country, makes this attack highly effective. The authors found
that the acceptance rate for cloned profiles was over 60\%. Much higher than the
acceptance rate of 30\% for fictitious profiles. Once the friendship is
established, it is possible to extract private information that is not available
publicly and perform identity theft.

The idea of cross-site profile cloning is similar, except the cloned profile is
created on another social network site that the victim does not yet use. Once
the cloned profile is created, iCloner attempts to identify friends of the
victim and begins sending friend requests. Similarly, 56\% of the friend
requests were accepted. These examples demonstrate that the carelessness of
users and the ability to create Sybils makes makes OSN vulnerable to identity
theft.

\subsubsection{Spam}
Spam \cite{facebookspam}

\subsubsection{Fake Friends?}

\subsection{The Sybil Attack in Reputation Systems}

\subsubsection{Reputation Systems}
\subsubsection{Attacks}

\subsection{The Sybil Attack in MANET}
% TODO should we include this?

\subsection{TODO}
a test bed for sybil attacks\cite{irissappane2012towards}

Quantifying Sybil attack\cite{margolin2008quantifying}

%%% Local Variables:
%%% mode: latex
%%% TeX-master: "main"
%%% End:
