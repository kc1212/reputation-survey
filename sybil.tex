The Sybil attack is coined by Douceur\cite{douceur2002sybil} in 2002 in the
context of peer-to-peer systems. In this section, we first introduce the Sybil
attack using Douceur's original definition and outline the key (discouraging)
theoretical results. Next, we review practical attacks in three types of systems
(1) MANET (mobile ad-hoc networks) such as sensor networks, (2) reputation
systems such as PageRank\cite{page1999pagerank} but also include e-commerce
systems such as eBay and (3) OSN (online social networks) such as Twitter and
Facebook. We hope our review illuminates the alarming consequences of the Sybil
attack.

\subsection{Theoretical Results}\label{sec:sybil-theory}
Douceur defined the Sybil attack as forging multiple identities under the same
entity\cite{douceur2002sybil}. An entity can be for example a physical user of
the system and identities are how entities present themselves to the system.
Thus a local entity has no direct knowledge of remote entities, only their
identities. The forged identities do not necessarily follow the protocol
specified by the underlying network, they may deviate arbitrarily from the
protocol, i.e. they assume the characteristics of Byzantine
fault\cite{lamport1982byzantine}. We use these terms in the remainder of the
survey.

The author modelled the system as a general distributed computing
environment where there is no constraint on the topology, every node has limited
computational resources and messages are guaranteed to be delivered. Under this
model, the author proved that the Sybil attack is always possible without a
central, trusted authority.

% Preventing the Sybil attack is in fact a lot
% more difficult because peer-to-peer systems often do not have a central, trusted
% authority.

Cheng and Friedman proved an important result regarding the Sybil attack in
reputation systems\cite{cheng2005sybilproof}. Reputation systems are commonly
used in MANET, e-commerce and the internet in general, where entities are
rewarded by their good behaviour and penalised otherwise. Google's
PageRank\cite{page1999pagerank} is an example of a reputation system, where a
large number of links to a website makes it more reputable. Cheng and Friedman
classified reputation systems into two categories,
\begin{enumerate}
\item symmetric reputation systems where the reputation score only depends on
  the network topology, popular reputation mechanisms such as
  PageRank\cite{page1999pagerank} and EigenTrust\cite{kamvar2003eigentrust} are
  examples of symmetric reputation systems, and
    \item asymmetric reputation systems where there some nodes are trusted and
      reputation scores are propagated through the trusted nodes, most OSN are
      examples of asymmetric reputation systems.
\end{enumerate}
The authors formally proved that symmetric reputation systems are vulnerable to
the Sybil attack. But in the asymmetric case, it is possible to construct a
Sybil-proof reputation system.

\subsection{The Sybil Attack in P2P FIle Sharing Networks}
P2P (peer-to-peer) file sharing networks are distributed computer networks that
are built for discovering and sharing files. BitTorrent\cite{bep3} is likely the
most popular P2P network at the time of writing. Due to their open and
distributed nature, they are vulnerable to the Sybil attack.

\subsubsection{Denial of Service}
By exploiting vulnerabilities in the BitTorrent network, denial of service
attack can be directed at any machine connected to the internet, not just
machines in the network\cite{sia2006ddos}. The main idea is to report the victim
as the tracker (a server that coordinates the peers). El Defrawy, Gjoka and
Markopoulou created a small scale proof-of-concept attack. Using only one
machine, they could generate enough traffic to cripple small organisations and
home users. The authors suggested that if Sybils are created to perform the same
attack aimed at a single victim, then it could easily throttle links with much
higher bandwidth\cite{el2007bottorrent}.

% Gnutella?
% \cite{daswani2002query}
% \cite{daswani2004pong}

\subsubsection{Index Poisoning}
P2P networks often implement a DHT (distributed hash table). The DHT in
BitTorrent is called Mainline-DHT, based on
Kademlia\cite{maymounkov2002kademlia}. Keys are the infohashes (file
identifiers) and values are the metadata of the files, these are distributed
across all the participating peers. Every node stores a routing table and
requests are routed iteratively to the node responsible for a particular
key\cite{bep5}. The goal of index poisoning is to corrupt routing table so that
honest peers fail to find the values they want. It can be mounted by injecting
Sybils into the DHT that do not follow the protocol. Wang and Kangasharju
created honeypots in the BitTorrent network and detected as many as 300,000
Sybils\cite{wang2012real}. Similar attacks are possible in other P2P networks
such as Overnet\cite{liang2006index}.

% There are no concrete proof of malicious activities, but the Sybils can easily
% sabotage the DHT if they choose to.

% 

\subsection{The Sybil Attack in Online Social Networks}
OSN (online social networks) are vulnerable to the Sybil attack even when most
of them use a central, trusted authority such as Facebook. In OSN, users create
profiles and form relationships with friends. In contrast with real world
relationships, it is much easier to create relationships in OSN even with
strangers. In 2008, Sophos conducted an experiment where they created a Facebook
profile and send friend requests to 200 random users, and 41\% of the users
accepted the friend request\cite{sophos}. A report by Facebook at the end of
2011 stated 5-6\% of their accounts are fake\cite{facebookfake}. Combining with
the ability to create new identities with very little cost, it is possible to
perform many types of attacks which we outline below.

\subsubsection{Identity Theft}
Authors of \cite{bilge2009all} created two attacks - profile cloning and
cross-site profile cloning, targeting five social network sites including
Facebook and LinkedIn. The iCloner system was created to automate these
attacks.

In profile cloning, iCloner uses publicly available information to automatically
create clones of the victim's profiles, effectively creating Sybils. iCloner
then sent friend requests from the cloned profile to the friends of the victim.
The fact that the victim may have many friends that they do not contact very
often, e.g. friend from primary school living in another country, makes this
attack highly effective. The authors found that the acceptance rate for cloned
profiles was over 60\%. Much higher than the acceptance rate of 30\% for
fictitious profiles. Once the friendship is established, it is possible to
extract private information that is not available publicly and perform identity
theft.

The idea of cross-site profile cloning is similar, except the cloned profile is
created on another social network site that the victim does not yet use. Once
the cloned profile is created, iCloner attempts to identify friends of the
victim and begins sending friend requests. Similarly, 56\% of the friend
requests were accepted. 

A more recent study created SbN (Socialbot Network) targeting
Facebook\cite{boshmaf2011socialbot}. Each socialbot is a Sybil created by the
attack, it controls a forged profile and minic human behaviour to avoid
detection. The attacker is the botmaster who coordinates the socialbots to
achieve a common objective such as infiltrating the target OSN by creating
friend relationships with real users. The authors found that infiltration
success rate was as high as 80\% and the FIS\cite{stein2011facebook} (Facebook
Immune System) was not sufficient to prevent the attack. Once the relationships
are established, the botmaster can command the socialbots to start gathering
private information which can then be used for identity theft.

% authors also said that Sybil detection based on attack edges is not effective
% because it's easy to create trust relationships with strangers

These examples demonstrate that the carelessness of users and the ability to
create Sybils makes OSN vulnerable to identity theft. Moreover, identity theft
is only an entry point. Once trust relationships are established, the attacker
can perform many other types of attacks such as spamming, phishing or
astroturfing to gain advantage.

\subsubsection{Astroturf}
Astroturfing is an act of creating grassroot movement that are in reality
carried out by a single entity, effectively spreading misinformation to
legitimate users. It relies on the ability to create Sybils in the underlying
social network. This type of attack is especially effective in social networks
such as Twitter where a lot of the social interaction such as sending messages
happen in the public.

In the 2010 Massachusetts senate race, Mustafaraj and Metaxas found evidence
that Republican campaingners created fake Twitter accounts and used them to send
spam. The spam caused Google real-time search results to tip in their favour
thus causing a spread of misinformation\cite{mustafaraj2010obscurity}.
Ratkiewicz et el. suggest that this type of attack can be mounted cheaply and
may have a larger influence than traditional
adversiting\cite{ratkiewicz2011truthy}.

The Truthy system\cite{ratkiewicz2011truthy} is a web service that perform
real-time analysis of Twitter to detect political astroturfing. In the 2010
U.S. midterm election, the authors found accounts which generated a lot of
retweets but no original tweets. More importantly, they uncovered a network of
bot accounts that injected thousands of tweets to smear the Democratic candidate.

In 2012, Wang et el. investigated two of the largeset
crowdturfing\footnote{Crowdsourced astroturfing.} platforms in China that brings
together buyers and sellers - Zhubajie and Sandaha. One of their services is
perform astroturfing on Weibo (The Chinese Twitter). The authors found that the
5364 sellers collectively own 14151 Weibo accounts and the top 1\% of the
sellers own over 100 accounts. Furthermore, the business is growing and more
than \$4 million have been spent on these two platforms over five
years\cite{wang2012serf}.


\subsubsection{Spam}
\cite{stringhini2010detecting}
\cite{gao2010detecting}
\cite{yang2012analyzing}

\subsection{The Sybil Attack in Reputation Systems}
Reputation systems cultivate collaborative behaviour by allowing entities to
trust each other based on community feedback, usually in the form of a
reputation score. Entities decide whom to trust based on the reputation scores,
thus entities are also incentivised to behave honestly. Reputation systems are
found in many context. In e-commerce, namely eBay, researchers found that the
merchant's reputation ``is a statistically and economically significant
determinant of auction prices''\cite{houser2006reputation}, and ``buyers are
willing to pay 8.1\% more'' for goods sold buy a reputable
merchant\cite{resnick2006value}. The file sharing peer-to-peer network
BitTorrent uses tit-for-tat as an ephemeral reputation system to encourage peers
to upload in exchange for better download speeds\cite{cohen2003incentives}. The
aforementioned PageRank\cite{page1999pagerank} is also a reputation system, used
for ranking reputable websites higher in Google's search results. Some OSN can
be considered as reputation systems too, Twitter accounts with a lot of
followers may be more reputable than accounts with fewer followers.

Unfortunately, reputation systems are also vulnerable to the Sybil attack.
Worryingly, there appears to an industry built around it, and their products are
easily accessible in the clearnet. In this section, we describe practical
attacks on reputation systems.

\subsubsection{Self-promoting}
In self-promotion, the goal of the attacker is to illegitimately raise its own
reputation. A common way to perform self-promition is to create Sybils and have
them create positive reputation for the attacker's main identity.

Dini and Spagnolo studied the economics of buying reputation on eBay. The
authors discovered many cheap items (around \euro{0.7}) for sell are simply there to
boost feedback. For example, one of the item is titled ``Apple Cranberry Crisp
Recipe + 100\% Positive Feedback''. The authors successfully boosted their
feedback by purchasing such items. But they made an unsuccessful attempted to
place a bid on their own good with a fake account\cite{dini2009buying}.

% Christin crawed the Silk Road, the anonymous marketplace. The author
% \cite{christin2013traveling}

De Cristofaro et el. performed an emperical study on Facebook page promotion
using like farms\cite{de2014paying}. Some of the farms such as
\verb!SocialFormulae.com! are clearly operated by bots and the operator does not
attempt to hide it, others such as \verb!BlostLikes.com! tries to minic human
users. The authors purchased the ``1000 likes'' service on their empty Facebook pages.
In under a month, many empty pages have accumulated almost 1000 likes as
promised by the like farms. The authors empty accounts were not terminated. Only
a small number of the liker's account were terminated.

% \cite{soska2015measuring}

SEOClerks and MyCheapJobs are also evidences of marketplaces for self-promotion.
Some of the top services include ``1 million Twitter followers'' at \$849,
``1000+ Instagram followers'' at \$10 and so on. The revenues of those two
marketplaces are estimated to be at \$1.3 million and \$116 thousand,
respectively\cite{farooqi2015characterizing}. Although the authors did not
investigate the properties of the fake followers, there is little doubt that
many of accounts used in these services are Sybils.

\subsubsection{Slandering}
The goal of a slandering attack is to illegitimately produce negative feedback
to undermine the reputation of the target. It is easy to imagine the improvement
in effectiveness when using multiple Sybils. From the best of our knowledge,
there are no published studies on real-world slandering. But
research has shown having a negative feedback may harm the target's ability to
do business\cite{ba2002evidence}.

\subsubsection{Whitewashing}
In whitewashing, attackers abuse the reputation system for temporary gain and
then escape the consequences by joining the reputation system under a new
identity to shed their bad reputation. Clearly, whitewashing is only possible
when the Sybil attack is possible. Again, there are no studies on whitewashing
in the real-world. But many have suggested that it is feasible
attack\cite{hoffman2009survey, marti2006taxonomy}.

\subsubsection{Denial of Service}
The denial of service attack highly depends on the structure of the reputation
system. 

\subsection{The Sybil Attack in MANET}
% TODO should we include this?

\subsection{TODO}
a test bed for sybil attacks\cite{irissappane2012towards}

Quantifying Sybil attack\cite{margolin2008quantifying}

%%% Local Variables:
%%% mode: latex
%%% TeX-master: "main"
%%% End:
