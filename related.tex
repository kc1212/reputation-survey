Many surveys on the Sybil attack exist in the literature. We attribute much of
the initial findings to these surveys. In contrast to the existing work,
we also try to cover the possible attacks that can be mounted using Sybils as
well as a wider range of defence mechanisms.

To the best of our knowledge, Levine et el. published the first survey of the
defence mechanisms\cite{marti2006taxonomy} in 2006. They found that the most
popular defence mechanism (in terms of the number of published work) at that
time is to use a certificate authority. The surveyed defence mechanisms
approximately cover sections \ref{sec:cert-authority},
\ref{sec:resource-testing} and \ref{sec:registration-fee} in this work.

Many years later, Mohaisen and Kim published a survey focusing on 

Reputation Surveys:
\cite{marti2006taxonomy}
\cite{josang2007survey} ?
\cite{hoffman2009survey}
\cite{koutrouli2012taxonomy}
\cite{selvaraj2012survey} ?
\cite{hendrikx2015reputation}

Sybil Surveys:
\cite{mohaisen2013sybil}
\cite{rakesh2014survey}
\cite{gunturu2015survey}
\cite{koll2014state}
Sok\cite{alvisi2013sok} but also some contribution

Other:
\cite{wallach2003survey}
DHT seucrity survey\cite{urdaneta2011survey}

%%% Local Variables:
%%% mode: latex
%%% TeX-master: "main"
%%% End:
