
Sybil attacks can be mounted in different applications and cause a large array
of consequences. This section categorises the attacks by the goal for four
common applications. (1) P2P (peer-to-peer) file sharing networks such as
BitTorrent, (2) OSN (online social networks) such as Twitter and Facebook, (3)
reputation systems such as eBay and (4) WANET (wireless ad-hoc networks) such as
sensor networks. We hope this section further illuminates the alarming
consequences of the Sybil attack.

\subsection{The Sybil Attack in File Sharing Networks}
File sharing networks are often P2P, where peers connect to other peers to
download and/or upload files without a central server (for the most part).
BitTorrent~\cite{bep3} is likely the most popular P2P network at the time of
writing. Unlike social networks where there is a trust relationship between
peers, P2P networks are much more open and distributed, making them much more
vulnerable to the Sybil attack.

P2P networks often implement a DHT (distributed hash table). The DHT in
BitTorrent is called Mainline-DHT, based on
Kademlia~\cite{maymounkov2002kademlia}. Keys are the infohashes (file
identifiers) and values are the metadata of the files, these are distributed
across all the participating peers. Every node also stores a routing table, the
entries in the table is percise for the closer nodes and coarse for node that
are further away. Requests to search for a key are routed iteratively, every
iteration the request gets closer to the destination node. Finally the value is
returned if it is found~\cite{bep5}.

\subsubsection{Index Poisoning}
\label{sec:index-poisoning}
The goal of index poisoning is to corrupt routing table so that
honest peers fail to find the values they want. It can be mounted by injecting
Sybils into the DHT that do not follow the protocol. Wang and Kangasharju
created honeypots in the BitTorrent network and detected as many as 300,000
Sybils in 2012~\cite{wang2012real}. Similar attacks are possible in other P2P networks
such as Overnet~\cite{liang2006index}.

% There are no concrete proof of malicious activities, but the Sybils can easily
% sabotage the DHT if they choose to.

% \subsubsection{Unfair Use of Resources}
% creating sybils with ``strategic'' missing pieces

% Gnutella?
% \cite{daswani2002query}
% \cite{daswani2004pong}
\subsubsection{Eclipse Attack}
Steiner et al. mounted an Eclipse attack on a Kademlia based DHT for P2P file
sharing known as KAD, it is used in P2P networks Overnet and
eMule~\cite{steiner2007exploiting}. The Eclipse Attack~\cite{singh2006eclipse}
is a special form of targeted Sybil attack. Sybils are arranged in the network
such that they \emph{eclipse} the the victim from the rest of the network. The
victim can either be an identity or an object such as a key in DHT. The authors
mounted the latter variation. The authors show that it is possible to eclipse a
key using only 32 Sybils in a DHT with 1.5 million users and 42,000 key.


\subsubsection{Denial of Service}
By using sybils and exploiting vulnerabilities in the BitTorrent network, denial
of service attack can be directed at any machine connected to the internet, not
just machines in the network~\cite{sia2006ddos}. The main idea is to report the
victim as the tracker (a server that coordinates the peers). El Defrawy, Gjoka
and Markopoulou created a small scale proof-of-concept attack. Using only one
machine, they could generate enough traffic to cripple small organisations and
home users. The authors suggested that if Sybils are created to perform the same
attack aimed at a single victim, then it could easily throttle links with much
higher bandwidth~\cite{el2007bottorrent}.

Steiner et al. also succeeded in mounting a DDoS attack but in the context of
the aforementioned KAD DHT~\cite{steiner2007exploiting}. Instead of replying the
correct list of peers to DHT queries, the Sybils always respond with the IP
address of the target peer in an attempt to overwhelm the target. The authors
show evidence that real-world malicious DDoS attacks involving more than 300,000
peers are mounted using P2P networks.

\subsubsection{Spying}
Many authors have used Sybils to monitor a P2P file sharing network that uses
DHT~\cite{holz2008measurements, steiner2007exploiting}, but such techniques can
be directly applied to spy on users. In essence, the authors created a lot of
light-weight sybils and tricked all the honest peers to store them in their
routing table, a form of index poisoning. The Sybils are light-weight because
they do not follow the DHT protocol and perform much simpler operations, a
single machine can have thousands of Sybils running simultaneously. Finally, DHT
requests would ``traverse through'' the Sybils due to the poisoned routing
table, and the requests are stored in a database for further analysis.

\subsection{The Sybil Attack in OSN}
OSN (online social networks) are vulnerable to the Sybil attack even when most
of them use a central, trusted authority. In OSN, users create
profiles and form relationships with friends. In contrast with real world
relationships, it is much easier to create relationships in OSN even with
strangers. In 2008, Sophos conducted an experiment where they created a Facebook
profile and send friend requests to 200 random users, and 41\% of the users
accepted the friend request~\cite{sophos}.

Many OSN in fact have a large number of Sybils. A report by Facebook at the end
of 2011 stated 5-6\% of their accounts are fake~\cite{facebookfake}. Jiang et el.
analysed data from Renren\footnote{One of the largest social network in China.},
they discovered 2440 sybil groups which totals to about 1 million sybils
\cite{jiang2015understanding}.

Attackers leverage the ability to fool users into becoming one of their Sybil's
friend and the ability to create a large number of Sybils to mount a large
variety of attacks on the user. We outline the different types of attack in this
section. Note that online social networks often have a reputation aspect as
well, for example a Facebook page with a lot of fans may be considered to be
more reputable than pages with a lower number of fans. We discuss attacks
specific to OSN in this section and attacks on reputation in
\autoref{sec:reputation-attack}

\subsubsection{Identity Theft}
Identity theft is the intentional user of the victim's identity to gain an
advantage. Authors of~\cite{bilge2009all} created two attacks - profile cloning
and cross-site profile cloning, targeting five social network sites including
Facebook and LinkedIn. The iCloner system was created to automate these attacks.

In profile cloning, iCloner uses publicly available information to automatically
create clones of the victim's profiles. iCloner then sents friend requests from
the cloned profile to the friends of the victim. The fact that the victim may
have many friends that they do not contact very often, e.g. friend from primary
school living in another country, makes this attack highly effective. The
authors found that the acceptance rate for cloned profiles was over 60\%. Much
higher than the acceptance rate of 30\% for fictitious profiles. Once the
friendship is established, it is possible to extract private information that is
not available publicly and perform identity theft.

The idea of cross-site profile cloning is similar, except the cloned profile is
created on another social network site that the victim does not yet use. Once
the profile is cloned, iCloner attempts to identify friends of the
victim and begins sending friend requests. Similarly, 56\% of the friend
requests were accepted. 

More recently, Boshmaf et el. created SbN (Socialbot Network), which targets
Facebook~\cite{boshmaf2011socialbot}. Each socialbot is a sybil created by the
attacker, it controls a forged profile and minics human behaviour to avoid
detection. The attacker is the botmaster who coordinates the socialbots to
achieve a common objective such as infiltrating the target OSN by creating
friend relationships with real users. The authors found that infiltration
success rate was as high as 80\% and the FIS (Facebook Immune
System~\cite{stein2011facebook}) was not sufficient to prevent the attack. Once
the relationships are established, the botmaster can command the socialbots to
start gathering private information which can then be used for identity theft.

% authors also said that Sybil detection based on attack edges is not effective
% because it's easy to create trust relationships with strangers

These examples demonstrate that the carelessness of users and the ability to
create Sybils makes OSN vulnerable to identity theft. Moreover, identity theft
is only an entry point. Once trust relationships are established, the attacker
can perform many other types of attacks such as spamming, phishing or
astroturfing to gain advantage.

\subsubsection{Astroturf}
Astroturfing is an act of creating grassroot movement that are in reality
carried out by a single entity, effectively spreading misinformation to
legitimate users. It relies on the ability to create Sybils in the underlying
social network. This type of attack is especially effective in social networks
such as Twitter where a lot of the social interaction such as sending messages
happen in the public.

In the 2010 Massachusetts senate race, Mustafaraj and Metaxas found evidence
that Republican campaingners created fake Twitter accounts and used them to send
spam. The spam caused Google real-time search results to tip in their favour
thus causing a spread of misinformation~\cite{mustafaraj2010obscurity}.
Ratkiewicz et el. suggest that this type of attack can be mounted cheaply and
may have a larger influence than traditional
adversiting~\cite{ratkiewicz2011truthy}.

The Truthy system~\cite{ratkiewicz2011truthy} is a web service that perform
real-time analysis of Twitter to detect political astroturfing. In the 2010
U.S. midterm election, the authors found accounts which generated a lot of
retweets but no original tweets. More importantly, they uncovered a network of
bot accounts that injected thousands of tweets to smear the Democratic candidate.

In 2012, Wang et el. investigated two of the largeset crowdturfing (crowdsourced
astroturfing) platforms in China --- Zhubajie and Sandaha. One of their services
is perform astroturfing on Weibo\footnote{The Chinese Twitter}. The authors
found that the 5364 sellers collectively own 14151 Weibo accounts and the top
1\% of the sellers own over 100 accounts. Furthermore, the business is growing
and more than \$4 million have been spent on these two platforms over five
years\cite{wang2012serf}.

\subsubsection{Spam}
Spamming, much like in the context of email, is the act of sending unsolicited
or undesire messages (spam). The goal of the attacker varies from advertisement
to phishing and spreading malicious software~\cite{twittermalware1,
  twittermalware2}. Many studies have characterised the behaviour of the
spammers and found that many spammers are in fact automated
sybils~\cite{stringhini2010detecting, yang2012analyzing, grier2010spam,
  jiang2015understanding}. More importantly, spamming is possibly the most
common attack in OSN. Jiang et el. analysed the malicious activities of the
sybils on Renren and found propogating advertisement (at 32.8\%) is the most
common one~\cite{jiang2015understanding}. Some authors have worked with the
service provide to close the spam accounts, but it is clearly not sufficient as
we described in \autoref{sec:motivation}.


\subsection{The Sybil Attack in Reputation Systems}
\label{sec:reputation-attack}
Reputation systems cultivate collaborative behaviour by allowing entities to
trust each other based on community feedback, usually in the form of a
reputation score. Entities decide whom to trust based on the reputation scores,
thus entities are also incentivised to behave honestly. Reputation systems are
found in many context. In e-commerce, namely eBay, researchers found that the
merchant's reputation ``is a statistically and economically significant
determinant of auction prices''~\cite{houser2006reputation}, and ``buyers are
willing to pay 8.1\% more'' for goods sold buy a reputable
merchant~\cite{resnick2006value}. The file sharing peer-to-peer network
BitTorrent uses tit-for-tat as an ephemeral reputation system to encourage peers
to upload in exchange for better download speeds~\cite{cohen2003incentives}.
PageRank~\cite{page1999pagerank} is also a reputation system, used for ranking
websites in Google's search results.

Reputation systems are also vulnerable to the Sybil attack. Worryingly, there
appears to an industry built around it, and their products are easily accessible
in the clearnet. In this section, we describe practical attacks on reputation
systems.

\subsubsection{Self-promoting}
In self-promotion, the goal of the attacker is to illegitimately raise its own
reputation. A common way to perform self-promition is to create Sybils and have
them create positive reputation for the attacker's main identity.

\begin{comment}
Dini and Spagnolo studied the economics of buying reputation on eBay. The
authors discovered many cheap items (around \euro{0.7}) for sell are simply there to
boost feedback. For example, one of the item is titled ``Apple Cranberry Crisp
Recipe + 100\% Positive Feedback''. The authors successfully boosted their
feedback by purchasing such items. But they made an unsuccessful attempted to
place a bid on their own good with a fake account~\cite{dini2009buying}.
\end{comment}

% Christin crawed the Silk Road, the anonymous marketplace. The author
% \cite{christin2013traveling}

De Cristofaro et el. performed an emperical study on Facebook page promotion
using like-farms~\cite{de2014paying}. Some of the farms such as
\verb!SocialFormulae.com! are clearly operated by bots and the operator does not
attempt to hide it, others such as \verb!BoostLikes.com! tries to minic human
users. The authors purchased the ``1000 likes'' service on their empty Facebook
pages. In under a month, many empty pages have accumulated almost 1000 likes as
promised by the like-farms. A month later, the authors empty accounts were not
terminated, only a small number of the liker's account were terminated.

% \cite{soska2015measuring}

SEOClerks and MyCheapJobs are also evidences of marketplaces for self-promotion.
Farooqi et el. found some of the top services include ``1 million Twitter
followers'' at \$849, ``1000+ Instagram followers'' at \$10 and so on. The
revenues of those two marketplaces are estimated to be at \$1.3 million and
\$116 thousand, respectively~\cite{farooqi2015characterizing}. Although the
authors did not investigate the properties of the fake followers, there is
little doubt that many of accounts used in these services are sybils.

\subsubsection{Slandering}
The goal of a slandering attack is to illegitimately produce negative feedback
to undermine the reputation of the target. It is easy to imagine the improvement
in effectiveness when using multiple Sybils. Many have suggested it is a
feasible attack~\cite{hoffman2009survey, koutrouli2012taxonomy}, but from the
best of our knowledge, there are no published studies on real-world slandering.
But research has shown having a negative feedback may harm the target's ability
to do business~\cite{ba2002evidence}.

\subsubsection{Whitewashing}
In whitewashing, attackers abuse the reputation system for temporary gain and
then escape the consequences by joining the reputation system under a new
identity to shed their bad reputation. Clearly, whitewashing is only possible
when the Sybil attack is possible. Again, there are no studies on whitewashing
in the real-world. But many have suggested that it is feasible
attack\cite{hoffman2009survey, marti2006taxonomy}.

% TODO
% \subsubsection{Denial of Service}
% The denial of service attack highly depends on the structure of the reputation
% system. 

\subsection{The Sybil Attack in WANET}
WANET (wireless ad-hoc networks) is a dynamic, self-configuring, self-healing
wireless network. Ad-hoc in this case means it does not rely on existing
infrastructure for the network to function. Each node in the network is
responsible for some general tasks such as routing, and some application
specific tasks such as gathering data from its sensors in the case of a sensor
network.

Akin to the other applications, an attacker in a WANET may own a single physical
node, but it may behave as if it were a large number of nodes. Many WANET
designs involve a reputation system\cite{ganeriwal2008reputation,
  buchegger2003robust}, thus the same attacks from
\autoref{sec:reputation-attack} applies here. In this section we describe the
WANET specific attacks. From the best of our knowledge WANET are not widely
deployed in practice, thus there is little research on real-world attacks.

\subsubsection{Unfair Resource Allocation}
Nodes in WANET often have limited resources such as bandwidth of the radio
channels. Resources such as these must be shared between the neighbours using
time slices. When the neighbours are Sybils, then the attacker can receive an
unfair amount of resource allocation and denies resources for the honest
nodes\cite{newsome2004sybil}. In contrast with the other attacks, this works
even when the Sybils are not behaving malicioiusly.

\subsubsection{Routing Disruption}
An important routing technique is multipath routing, data is routed using
multiple paths in the network for better fault-tolerance and bandwidth. However,
if Sybils are present in the network, then the different paths may in fact go
through the Sybils owned by a single attacker, nullifying the advantages.
Another technique is geographic routing, nodes route data depending on the
geographic location of their neighbours. Sybils in the network can pretend to be
in more than one place at a time, thus significant distrupting the routing
algorithm\cite{karlof2003secure}.

\subsubsection{Spreading False Information}
Nodes often need to exchange information with each other to satisfy the
underlying requirements of the application. Some of the common tasks are data
aggregation and voting. With enough Sybils, it is possible to manipulate the
aggregated data or the poll to benefit the attacker. For example, sensor
networks may use a bollot to detect misbehaving nodes, the attack could use its
Sybils to claim that a honest node is misbehaving and have the other nodes expel
it from the network\cite{newsome2004sybil}.

%%% Local Variables:
%%% mode: latex
%%% TeX-master: "main"
%%% End:
