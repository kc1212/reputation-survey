The sybil attack can be seen as an umbrella term for all attacks that require
the use of sybils. This section categorises the derivatives of the sybil attack
by the attacker's aim. Some attacks are more general, such as spamming, others
are application specific, i.e. astroturfing. The attacks are sorted by their
generality in decreasing order. We hope this section further illuminates the
alarming consequences of the sybil attack.

\subsection{Spamming}
Spamming is the act of sending unsolicited or undesired messages (spam). The
goal of the attacker varies from advertisement to phishing and spreading
malicious software~\cite{twittermalware1, twittermalware2}. Although spam can
originate from a single account, but it is much more difficul to detect and
prevent when it is from a group of sybils. In the context of email, spam can be
prevented in most cases when a large service provide is involved, such as
Gmail~\cite{adwords}. The same is not true in other applications.

For social networks, many studies have characterised the behaviour of the
spammers and found that many spammers are in fact automated
sybils~\cite{stringhini2010detecting, yang2012analyzing, grier2010spam,
  jiang2015understanding}. More importantly, spamming is possibly the most
common attack. Jiang et el. analysed the malicious activities of the sybils on
Renren\footnote{One of the largest social network in China
  (\texttt{http://renren.com/}).} and found propagating advertisement (at
32.8\%) is the most common one~\cite{jiang2015understanding}. Some authors have
worked with the service provide to close the spam accounts, but it is clearly
not sufficient as we described in \autoref{sec:motivation}.

In a very different application---cryptocurrencies, the spammer's goal is to
waste system resources rather than advertisements such as in social networks.
Block sizes in Bitcoin is fixed to 1 MB and block generation takes about 10
minutes. Thus, there is an upper bound for the number of transactions that the
Bitcoin network can handle per second (approximately 7 at the time of
writing~\cite{bitcointps}). Flooding the network with useless transactions will
cause its performance to drop. Decreasing the scalability even more.
Fortunately, Bitcoin incorporates a transaction fee, so spammers cannot abuse
the network indefinitely~\cite{bitcoinspam}.


\subsection{Slandering}
The goal of a slandering attack is to illegitimately produce negative feedback
to undermine the reputation of the target. It is mostly found in reputation
systems where users are rewarded for their positive behaviour and penalised for
their bad behaviour. The attack is difficult to prevent when the feedback is not
authenticated. Furthermore, an attacker who uses a group of sybils can make the
attack much more effective~\cite{hoffman2009survey}.

Mike Hearn suggests that Google has observed the attack in AdWords. The
attackers would submit false reports on AdWords advertisements to give them a
bad reputation and gain an advantage over their competitors~\cite{adwords}.
% Having a negative feedback may harm the target's ability to do
% business~\cite{ba2002evidence}.

Slandering also exist in wireless ad-hoc networks. Nodes in the network need to
exchange information with each other to satisfy the underlying requirements of
the application. Some common tasks are data aggregation and voting. With enough
sybils, it is possible to manipulate the aggregated data or the poll to benefit
the attacker. For example, sensor networks may use a ballot to detect
misbehaving nodes, the attack could use its sybils to claim that an honest node
is misbehaving and have the other nodes expel it from the
network~\cite{newsome2004sybil}.

\subsection{Self-promoting}
In self-promotion, the goal of the attacker is to illegitimately raise its own
reputation. It is also found in reputation systems, but it does not need to have
penalisation property (e.g. the number of likes on a Facebook page or the number
of followers on a Twitter account). A common way to perform self-promotion is to
create sybils and have them create positive reputation for the attacker's main
identity.

\begin{comment}
Dini and Spagnolo studied the economics of buying reputation on eBay. The
authors discovered many cheap items (around \euro{0.7}) for sell are simply there to
boost feedback. For example, one of the item is titled ``Apple Cranberry Crisp
Recipe + 100\% Positive Feedback''. The authors successfully boosted their
feedback by purchasing such items. But they made an unsuccessful attempted to
place a bid on their own good with a fake account~\cite{dini2009buying}.
\end{comment}

% Christin crawed the Silk Road, the anonymous marketplace. The author
% \cite{christin2013traveling}

De Cristofaro et el. performed an empirical study on Facebook page promotion
using like-farms~\cite{de2014paying}. Some farms such as
\verb!SocialFormulae.com! are clearly operated by bots and the operator does not
attempt to hide it, others such as \verb!BoostLikes.com! tries to mimic human
users. The authors purchased the ``1000 likes'' service on their empty Facebook
pages. In under a month, many empty pages have accumulated almost 1000 likes as
promised by the like-farms. A month later, the authors empty accounts were not
terminated, only a few of the liker's account were terminated.

% \cite{soska2015measuring}

SEOClerks and MyCheapJobs are also evidences of marketplaces for self-promotion.
Farooqi et el. found some of the top services include ``1 million Twitter
followers'' at \$849, ``1000+ Instagram followers'' at \$10 and so on. The
revenues of those two marketplaces are estimated to be at \$1.3 million and
\$116 thousand, respectively~\cite{farooqi2015characterizing}. Although the
authors did not investigate the properties of the fake followers, there is
little doubt that many of accounts used in these services are sybils.

In the same vein as the slandering attack, wireless ad-hoc networks also suffer
from the self-promoting attack. Nodes in the network often have limited
resources such as bandwidth of the radio channels. Resources such as these must
be shared between the neighbours using time slices. When the neighbours are
sybils, then the attacker can receive an unfair amount of resource allocation,
giving it a much higher privilege~\cite{newsome2004sybil}. This works even when
the sybils are themselves following the protocol.

\subsection{Whitewashing}
Whitewashing is when attackers abuse the reputation system for temporary gain
and then escape the consequences by joining the reputation system under a new
identity to shed their bad reputation~\cite{marti2006taxonomy}. For example, a
Reddit user may delete his or her account when it accumulates a lot of negative
karma, and then create a new account. The attack can be performed very easily if
it is cheap to create sybils~\cite{hoffman2009survey}.

Whitewashing is a problem in file sharing networks that involve a central
component to act as a reputation system. To prevent and penalise selfish
behaviour such as free-riding, a reputation system must be used. But many
studies have found that the reputation system in fact encourages
whitewashing~\cite{feldman2004free, yang2005empirical}.

\subsection{Routing Disruption}\label{sec:routing-disruption}
The aim of the attacker in routing disruption is to stop honest nodes from
sending or receiving the correct data by maliciously modifying its intermediate
routes.

In file sharing DHTs, routing disruption is commonly accomplished by index
poisoning. That is when the attacker corrupts the routing tables so that honest
peers fail to find the values they want. It can be mounted by injecting sybils
into the DHT that do not follow the protocol. Wang and Kangasharju created
honeypots in the BitTorrent network and detected as many as 300,000 sybils may
be performing such an attack~\cite{wang2012real}. Similar attacks are possible
in other P2P networks such as Overnet~\cite{liang2006index}.

In wireless ad-hoc networks, an important routing technique is multipath
routing, where data is routed using multiple paths in the network for better
fault-tolerance and bandwidth. However, if sybils are present in the network,
then the different paths may in fact go through the sybils owned by a single
attacker, nullifying the advantages. Another technique is geographic routing,
nodes route data depending on the geographic location of their neighbours.
Sybils in the network can pretend to be in more than one place at a time, thus
significant disrupting the routing algorithm\cite{karlof2003secure}.

\subsection{Eclipse Attack}
In the eclipse attack, sybils are arranged in the network such that they
\emph{eclipse} the victim from the rest of the network. The victim can either be
an identity or an object such as a key in a DHT (distributed hash
table)~\cite{singh2006eclipse}.

Steiner et al. mounted an Eclipse attack on the KAD DHT, it is used in P2P file
sharing networks Overnet and eMule~\cite{steiner2007exploiting}. The sybils are
deliberately crafted such that they automatically position themselves as the
neighbours of the target. The authors show it is possible to eclipse a key using
only 32 sybils in the DHT with 1.5 million users and 42,000 key.

Attacker can eclipse Bitcoin nodes to gain various unfair
advantages~\cite{heilman2015eclipse}. Bitcoin has no authentication, so nodes
receive unsolicited messages about the IP addresses of other nodes, these
addresses are stored in an internal table. The attacker essentially sends bogus
messages to the target that only contains the IP address of the attacker's
sybils, or ``rubbish'' addresses. This technique eclipses the target because it
can only connect to the attacker's IP address. This causes a lot of
consequences. (1) Engineering block races, when two nodes discover the next
block at the same time, the eclipsed block will not be able to receive the
reward. (2) Eclipsing a large portion of the network can cause a 51\% attack,
where a single party gain complete control of the network. (3) Selfish-mining is
when the attacker do not publish the latest block immediately after discovery
but aims to publish 2 or more blocks at the same time to gain a lead; this
results in many orphaned blocks for the other miners and the attacker can use
the eclipse attack to drop newly discovered blocks by other miners to make the
attack more effective. (4) Double spending, the eclipse attack allows the
attacker to double spend his Bitcoins to eclipse nodes because they don't have
an accurate view of the whole network.

\subsection{Denial of Service}
DoS (denial of service) attack is where the attacker seeks to deny the target
from doing work for its intended users. A common type of DoS is DDoS
(distributed denial of service), where the attacker makes use of multiple
identities to mount the attack.

In the BitTorrent network, DoS attack can be directed at any machine connected
to the internet, not just machines. It uses a vulnerability in the network and a
lot of sybils. The main idea is to report the victim as the tracker (a server
that coordinates the peers) such that peers that are looking for a particular
file would all query the victim~\cite{sia2006ddos}

El Defrawy, Gjoka and Markopoulou created a small scale proof-of-concept attack.
Using only one machine, they could generate enough traffic to cripple small
organisations and home users. The authors suggested that if sybils are created
to perform the same attack aimed at a single victim, then it could easily
throttle links with much higher bandwidth~\cite{el2007bottorrent}.

Steiner et al. also succeeded in mounting a DDoS attack but in the context of
the aforementioned KAD DHT~\cite{steiner2007exploiting}. Instead of replying the
correct list of peers to DHT queries, the sybils always respond with the IP
address of the target peer in an attempt to overwhelm the target. The authors
show evidence that real-world malicious DDoS attacks involving more than 300,000
peers are mounted using P2P networks.

\subsection{Eavesdropping}
Eavesdropping is the act of secretly recording the activities of one or more
nodes. It is made harder by the nature of distributed systems, because there is
no longer a central point of control. But sybils can be useful to overcome this
problem.

Many authors have used sybils to monitor a P2P file sharing network that uses
DHT~\cite{holz2008measurements, steiner2007exploiting}, such techniques can also
be directly applied to eavesdrop users. In essence, the authors created a lot of
light-weight sybils and tricked all the honest peers to store their addresses in
their routing table, a form of index poisoning. The sybils are light-weight
because they do not follow the DHT protocol and perform much simpler operations.
A single machine can have thousands of sybils running simultaneously. Finally,
DHT requests would ``traverse through'' the sybils due to the poisoned routing
table, and the requests are stored in a database for further analysis.

\subsection{Identity Theft}
Identity theft is the intentional use of the victim's identity to gain an
advantage. Authors of~\cite{bilge2009all} created two attacks - profile cloning
and cross-site profile cloning, targeting five social network sites including
Facebook and LinkedIn. The iCloner system was created to automate these attacks.

In profile cloning, iCloner uses publicly available information to automatically
create clones of the victim's profiles. iCloner then sends friend requests from
the cloned profile to the friends of the victim. The fact that the victim may
have many friends that they do not contact very often, e.g. friend from primary
school living in another country, makes this attack highly effective. The
authors found that the acceptance rate for cloned profiles was over 60\%. Much
higher than the acceptance rate of 30\% for fictitious profiles. Once the
friendship is established, it is possible to extract private information that is
not available publicly and perform identity theft.

The idea of cross-site profile cloning is similar, except the cloned profile is
created on another social network site that the victim does not yet use. Once
the profile is cloned, iCloner attempts to identify friends of the
victim and begins sending friend requests. Similarly, 56\% of the friend
requests were accepted. 

More recently, Boshmaf et el. created SbN (Socialbot Network), which targets
Facebook~\cite{boshmaf2011socialbot}. Each socialbot is a sybil created by the
attacker, it controls a forged profile and mimics human behaviour to avoid
detection. The attacker is the botmaster who coordinates the socialbots to
achieve a common objective such as infiltrating the target social network by creating
friend relationships with real users. The authors found that infiltration
success rate was as high as 80\% and the FIS (Facebook Immune
System~\cite{stein2011facebook}) was not sufficient to prevent the attack. Once
the relationships are established, the botmaster can command the socialbots to
start gathering private information which can then be used for identity theft.

% authors also said that Sybil detection based on attack edges is not effective
% because it's easy to create trust relationships with strangers

These examples demonstrate that the carelessness of users and the ability to
create sybils makes social networks vulnerable to identity theft. Moreover,
identity theft is only an entry point. Once trust relationships are established,
the attacker can perform many other types of attacks such as spamming, phishing
or astroturfing to gain advantage.

\subsection{Astroturfing}
Astroturfing is an act of creating grassroot movement that are in reality
carried out by a single entity, effectively spreading misinformation to
legitimate users. It relies on the ability to create sybils in the underlying
social network. This type of attack is especially effective in social networks
such as Twitter where a lot of the social interaction such as sending messages
happen in the public.

In the 2010 Massachusetts senate race, Mustafaraj and Metaxas found evidence
that Republican campaigners created fake Twitter accounts and used them to send
spam. The spam caused Google real-time search results to tip in their favour
thus causing a spread of misinformation~\cite{mustafaraj2010obscurity}.
Ratkiewicz et el. suggest that this type of attack can be mounted cheaply and
may have a larger influence than traditional
advertising~\cite{ratkiewicz2011truthy}.

The Truthy system~\cite{ratkiewicz2011truthy} is a web service that perform
real-time analysis of Twitter to detect political astroturfing. In the 2010
U.S. midterm election, the authors found accounts which generated a lot of
retweets but no original tweets. More importantly, they uncovered a network of
bot accounts that injected thousands of tweets to smear the Democratic candidate.

In 2012, Wang et el. investigated two of the largest crowdturfing (crowdsourced
astroturfing) platforms in China---Zhubajie and Sandaha. One of their services
is to perform astroturfing on Weibo\footnote{The Chinese Twitter
  (\texttt{http://weibo.com/}.)}. The authors found that the 5364 sellers
collectively own 14151 Weibo accounts and the top 1\% of the sellers own over
100 accounts. Furthermore, the business is growing and more than \$4 million
have been spent on these two platforms over five years\cite{wang2012serf}.

\begin{comment}
\subsection{The Sybil Attack in OSN}
OSN (online social networks) are vulnerable to the sybil attack even when most
of them use a central, trusted authority. Users create
profiles and form relationships with friends. In contrast with real world
relationships, it is much easier to create relationships in OSN even with
strangers. In 2008, Sophos conducted an experiment where they created a Facebook
profile and send friend requests to 200 random users, and 41\% of the users
accepted the friend request~\cite{sophos}.

Many OSN in fact have a large number of sybils. A report by Facebook at the end
of 2011 stated 5-6\% of their accounts are fake~\cite{facebookfake}. Jiang et
el. analysed data from Renren, they discovered 2440 sybil groups which
totals to about 1 million sybils \cite{jiang2015understanding}.

Attackers leverage the ability to fool users into becoming one of their sybil's
friend and the ability to create a large number of sybils to mount a large
variety of attacks on the user. We outline the different types of attack in this
section. Note that online social networks often have a reputation aspect as
well, for example a Facebook page with a lot of fans may be considered to be
more reputable than pages with a lower number of fans. We discuss attacks
specific to OSN in this section and attacks on reputation in
\autoref{sec:reputation-attack}



\subsection{The Sybil Attack in File Sharing Networks}
File sharing networks are often P2P, where peers connect to other peers to
download and/or upload files without a central server (for the most part).
BitTorrent~\cite{bep3} is likely the most popular P2P network at the time of
writing. Unlike social networks where there is a trust relationship between
peers, P2P networks are much more open and distributed, making them much more
vulnerable to the sybil attack.

P2P networks often implement a DHT (distributed hash table). The DHT in
BitTorrent is called Mainline-DHT, based on
Kademlia~\cite{maymounkov2002kademlia}. Keys are the infohashes (file
identifiers) and values are the metadata of the files, these are distributed
across all the participating peers. Every node also stores a routing table, the
entries in the table is precise for the closer nodes and coarse for node that
are further away. Requests to search for a key are routed iteratively, every
iteration the request gets closer to the destination node. Finally, the value is
returned if it is found~\cite{bep5}.


\subsection{The Sybil Attack in Reputation Systems}
\label{sec:reputation-attack}
Reputation systems cultivate collaborative behaviour by allowing entities to
trust each other based on community feedback, usually in the form of a
reputation score. Entities decide whom to trust based on the reputation scores,
thus entities are also incentivised to behave honestly. Reputation systems are
found in many contexts. In e-commerce, namely eBay, researchers found that the
merchant's reputation ``is a statistically and economically significant
determinant of auction prices''~\cite{houser2006reputation}, and ``buyers are
willing to pay 8.1\% more'' for goods sold buy a reputable
merchant~\cite{resnick2006value}. The file sharing peer-to-peer network
BitTorrent uses tit-for-tat as an ephemeral reputation system to encourage peers
to upload in exchange for better download speeds~\cite{cohen2003incentives}.
PageRank~\cite{page1999pagerank} is also a reputation system, used for ranking
websites in Google's search results.

Reputation systems are also vulnerable to the sybil attack. Worryingly, there
appears to an industry built around it, and their products are easily accessible
in the clearnet. In this section, we describe practical attacks on reputation
systems.
% TODO
% \subsubsection{Denial of Service}
% The denial of service attack highly depends on the structure of the reputation
% system. 

\subsection{The Sybil Attack in the Bitcoin Network}
Bitcoin is a cryptocurrency. It uses a global ledger that is replicated across
all nodes (miners). The ledger is created using blocks, chained together using
hash pointers to create a notion of ordering. Each block contains transactions.
Miners reach consensus using \emph{proof-of-work}~\cite{nakamoto2008bitcoin}.
This section describes how the Bitcoin network suffers from the sybil attack.
The consequences are fairly general and may apply to other blockchain networks
that use a global ledger.

% \subsubsection{Denial of Service}
% TODO~\cite{vasek2014empirical}

\subsection{The Sybil Attack in WANET}
WANET (wireless ad-hoc networks) is a dynamic, self-configuring, self-healing
wireless network. Ad-hoc in this case means it does not rely on existing
infrastructure for the network to function. Each node in the network is
responsible for some general tasks such as routing, and some application
specific tasks such as gathering data from its sensors in the case of a sensor
network.

Akin to the other applications, an attacker in a WANET may own a single physical
node, but it may behave as if it were many nodes. Many WANET designs involve a
reputation system\cite{ganeriwal2008reputation, buchegger2003robust}, thus the
same attacks from \autoref{sec:reputation-attack} applies here. In this section
we describe the WANET specific attacks. From the best of our knowledge WANET are
not widely deployed in practice, thus there is little research on real-world
attacks.
\end{comment}

%%% Local Variables:
%%% mode: latex
%%% TeX-master: "main"
%%% End:
